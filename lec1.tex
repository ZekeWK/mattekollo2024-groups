\documentclass[11pt,fleqn]{book} % Default font size and left-justified equations

\input{structure.tex} % Insert the commands.tex file which contains the majority of the structure behind the template

\usepackage{etoolbox}
\makeatletter
\patchcmd{\chapter}{\if@openright\cleardoublepage\else\clearpage\fi}{}{}{}
\makeatother
\pagestyle{empty}
\setcounter{chapter}{0}

\begin{document}
  \renewcommand*\rmdefault{ppl}\normalfont\upshape

\chapter{Gruppteori 1. Introduktion}
\kapiteldatum{20 juli}
\large
\thispagestyle{empty}

% TODO! Add motivating problems for all lessons

\begin{definition} En \textit{grupp} \(G\) är en mängd \(M\) tillsammans med en operator \(\star \) så att:
  \begin{enumerate} % Omformulera!
    \item För alla två element \(a, b\) i \(M\) definierar \( \star \) ett element \(a  \star b\) som också är i \(M\).
    \item För alla tre element \(a, b, c\) i \(M\) är \((a  \star b)  \star c = a  \star (b  \star c)\).
    \item Det finns ett \textit{identitets element} \(e\) så att för alla \(a\) i \(M\) så är \(a  \star e = e  \star a = a\).
    \item Varje element \(a\) i \(M\) har en \textit{invers}  \(a ^{-1}\) så att \(a  \star a^{-1} = a^{-1}  \star a = e\).
  \end{enumerate}
\end{definition}

% Annat exempel är flätor!
\begin{problem} Motivera att kraven för en grupp uppnås av
  \textbf{(a)} heltalen under addition \textbf{(b)} rotationer av en hexagon som bevarar den under sammansättning \textbf{(c)} drag på en Rubiks kub under sammansättning.
\end{problem}

\begin{problem} Motivera att kraven på en grupp \textbf{inte} uppnås av 
  \textbf{(a)} udda heltalen under addition \textbf{(b)} heltalen under multiplikation \textbf{(c)} heltalen under operationen \(a  \star b = a  \cdot b + 1\).
\end{problem}

\begin{problem}
  Mängden \(M\) består av ``klä om strumpan''-operationer:
  \begin{enumerate}
    \item Lämna allt som det är.
    \item Ta av och klä på andra foten.
    \item Ta av, vänd ut och in och klä på samma fot.
    \item Ta av, vänd ut och in och klä på andra foten.
  \end{enumerate}
  Från början har man en strumpa på en av fötterna och man har två fötter. Visa att \(M\) under sammansättning av operationerna är en grupp!
\end{problem}

\begin{problem}
  Låt mängden \(M = \left\{e, a, b\right\}\) och operatorn \(\star\) uppfylla multiplikationstabellen \\
  { \center
  \begin{tabular}{c|ccc}
    \( \star \) & \(e\) & \(a\) & \(b\) \\ \midrule
    \(e\) & \(e\) & \(a\) & \(b\) \\
    \(a\) & \(a\) & \(e\) & \(b\) \\
    \(b\) & \(b\) & \(a\) & \(e\) \\
  \end{tabular} \\ }
  Är \(G = (M, \star )\) en grupp?
\end{problem}


\begin{problem} \textbf{(Dihedrala grupper).} Den \textit{dihedrala} gruppen \(D_n\) är alla rotationer och speglingar av en regelbunden \(n\)-hörning som avbildar den på sig själv. I den dihedrala gruppen \(D_4\):
  \begin{enumerate}[label=\textbf{(\alph*)}]
    \item Vilka är de fyra rotationerna i gruppen?
    \item Vilka är de fyra speglingarna?
    \item Vilket är identitetselementet?
  \end{enumerate}
\end{problem}

\begin{problem} \textbf{(Dihedrala grupper).} Skapa en multiplikationstabell för elementen i \(D_3\)! % För den dihedrala gruppen \(D_3\) låt \(S_1, S_2, S_3\) representera de 3 speglingarna (bestäm i vilken ordning) och \(R_0, R_1, R_2\) de tre rotationerna (\(0, 120^\circ, 240 ^\circ \) medurs). Skapa en multiplikationstabell av elementen! Är \(D_3\) kommutativ?
\end{problem}

\begin{definition}
  Låt \(a\) vara ett element i en grupp och \(i\) ett heltal. Då är
  \[
    a ^ i =
    \begin{cases}
      \underbrace{a  \star \dots  \star a}_{i \text{ gånger} } &\text{ om } i > 0 \\
      \underbrace{a ^{-1}  \star \dots  \star a ^{-1}}_{i \text{ gånger} } &\text{ om } i < 0 \\
      e &\text{ om } i = 0
    \end{cases} 
  \]
\end{definition}

\begin{problem} \textbf{(Dihedrala grupper).}
  Låt \(S\) vara en bestämd spegling och \(R\) vara den ``kortaste'' rotationen medurs i \(D_n\). Kan alla element i \(D_n\) skrivas på antingen formen \(R ^i \) eller \(S  \cdot R ^{i}\) där \(i\) är ett heltal?
  % Kan man beskriva alla element som rotationen \(R\) opererad med sig själv ett antal gånger följt av noll eller en speglingar \(S\). Till exempel \(R  \star R  \star R S\), \(R  \star S\) eller \(R  \star R\)?
\end{problem}

\begin{problem} Låt \(a, b\) vara element i en grupp. Visa att \((a  \star b) ^{-1} = b ^{-1} \star a ^{-1}\).
\end{problem}

\begin{problem} \textbf{(Tavla).}
 Vi vill hänga en tavla med ett snöre (vars ändar är i tavlan) på två spikar. Till skillnad från vanligt, vill vi att om man tar bort någon spik (oavsett vilken) så ska tavlan ramla! Hur gör vi detta? % TODO Add picture
\end{problem}

\begin{problem} \textbf{(Tavla).}
  Hur gör vi om tavlan ska hänga på \(n\) spikar så att tavlan ramlar oavsett vilken spik vi tar bort?
\end{problem}

% \begin{problem}
%   I ett glasstånd säljs glassar av smaken vanlij, jordgubb och choklad. Vissa av de som står i kön har en glass av en smak de inte vill ha, och andra vill verkligen ha en glass av en viss smak! Om en person som har en glass, och en som inte vill ha sin står efter varanndra så kommer ett glassbyte att ske och båda går därifrån. Ge ett exempel på en kö där alla smaker är representerade, och där om alla som vill ha ett 
% 
%   Ge ett exempel på en glasskö där alla smaker är representerade, så att om någon glassmak skulle bli 
%   På stranden finns det några personer som har en glass
%   av antingen smaken jordgubb, choklad eller vanlij, men vill inte ha den
% \end{problem}


\begin{problem}
  Visa att om \(a, b, c\) är element i en grupp så innebär \(a  \star b = a  \star c\) att \(b = c\).
\end{problem}

\begin{problem} \textbf{(Rubiks kub).}
  Visa att repeterad applicering av ett godtyckligt drag på en Rubiks kub kommer leda tillbaks till var den började.
\end{problem}

\begin{problem} \textbf{(Rubiks kub).}
  Visa att inget drag på en Rubiks kub, om det appliceras repeterat, kommer att gå igenom varje möjlig position.
\end{problem}

\begin{problem} \textbf{(Kortlek).}
  Kan man göra samma slutsatser som vi gjorde om dragen på en Rubiks kuben på en blandning av en kortlek?
\end{problem}

\subsection*{Extrauppgifter}
\begin{problem} % Fördela om?
  Visa att identitetselementet är unikt.
\end{problem}

\begin{problem}
  Visa att varje element har exakt en invers.
\end{problem}

\begin{problem}
  Visa att snittet av två grupper är en grupp.
\end{problem}

\begin{problem}
  Hitta två grupper av samma storlek som ``beter sig'' annorlunda.
\end{problem}


% \begin{problem}
%   Det minsta positiva heltal \(n\) så att \(a ^ n = e\) för något \(a\) i en grupp kallas för ordningen av \(a\). Visa att ordningen om \(a^k = e\) måste \(n \) dela \(k\).
% \end{problem}


\end{document}
