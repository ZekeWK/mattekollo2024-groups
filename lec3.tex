\documentclass[11pt,fleqn]{book} % Default font size and left-justified equations

\input{structure.tex} % Insert the commands.tex file which contains the majority of the structure behind the template

\usepackage{etoolbox}
\makeatletter
\patchcmd{\chapter}{\if@openright\cleardoublepage\else\clearpage\fi}{}{}{}
\makeatother
\pagestyle{empty}
\setcounter{chapter}{2}

\begin{document}
\renewcommand*\rmdefault{ppl}\normalfont\upshape

\chapter{Gruppteori 3. Delgrupper \& sidoklasser}
\kapiteldatum{22 juli}
\large
\thispagestyle{empty}

\begin{problem}
  Hitta en delgrupp av permutationsgruppen \(S_4\) som är isomorf med ``klä om strumpan''-gruppen.
\end{problem}

\begin{problem} \label{problem:multiplikationstabellen}
  Gruppen \(G\) består av mängden \(\left\{e, \pi , \tau\right\}\) och operatorn \(\star\) som uppfyller multiplikationstabellen 
  { \center
  \begin{tabular}{c|cccc}
    \( \star \) & \(e\) & \(\pi \) & \(\tau\) \\ \midrule
    \(e\) & \(e\) & \(\pi\) & \(\tau\) \\
    \(\pi\) & \(\pi \) & \(\tau\) & \(e\) \\
    \(\tau\) & \(\tau \) & \(e\) & \(\pi\) \\
  \end{tabular}\\}
  \vspace{2ex}
  \noindent Hitta en isomorfi från \(G\) till en delgrupp av \(S_3\). % TODO! Double check
\end{problem}

\begin{problem}
	Hitta en isomorfi från den dihedrala gruppen \(D_4\) till en delgrupp av permutationsgruppen \(S_8\).
\end{problem}

\begin{theoremeT} \textbf{(Cayleys sats).} Varje grupp är isomorf med en delgrupp till någon permutationsgrupp \(S_n\).
\end{theoremeT}

\begin{problem}
  Låt \(G\) vara gruppen i problem \ref{problem:multiplikationstabellen}. Visa att \(f(x) = \pi\) permuterar elementen i \(G\).
\end{problem}

\begin{problem}
	Bevisa Cayleys sats konstruktivt.
\end{problem}

\begin{problem}
	Hitta alla grupper av storlek \textbf{(a)} \(3\) \textbf{(b)} \(4\).
\end{problem}

\begin{definition} Låt \(H = \left\{h_1, h_2, \dots, h_n\right\}\) vara en delgrupp av \(G\). För varje element \(a\) i \(G\) definierar vi \textit{sidoklassen} \(a \star H = \left\{a  \star h_1, a \star h_2, \dots, a \star h_n\right\}\).
\end{definition}

\begin{problem}
  Vilka är sidoklasserna till delgruppen av rotationer i den dihedrala gruppen \(D_4\)?
\end{problem}

\begin{problem}
  Vilka är sidoklasserna till delgruppen av \((\left\{e, S\right\},  \star \) i den dihedrala gruppen \(D_4\) om \(S\) är en specifik spegling?
\end{problem}

\begin{problem}
  Observera gruppen av heltal under addition. Vilka är sidoklasserna till delgruppen av tal delbara med 7?
\end{problem}

\begin{problem}
  Låt \(M\) vara en delgrupp till \(G\) med \(k\) element. Hur många element har \(a  \star M\)?
\end{problem}

\begin{problem} 
	Finns det något element som inte är i någon sidoklass?
\end{problem}

\begin{problem}
	Visa att om två sidoklasser till samma delgrupp båda innehåller ett visst element så sammanfaller de.
\end{problem}

\begin{problem} \textbf{(Lagranges sats).} 
	Visa att för en ändlig grupp \(G\) och en delgrupp \(H\) måste storleken av \(H\) dela storleken av \(G\).
\end{problem}

% \subsection*{Extrauppgifter}
% \begin{definition} \textbf{(Normal delgrupp).} 
%   En \textit{normal} delgrupp \(H\) till gruppen \(G\) uppfyller att \(aH = Ha\) för alla \(a\) i \(G\).
% \end{definition}
% 
% \begin{problem}
%   Är mängden av alla \textbf{(a)} speglingar \textbf{(b)} rotationer en normal delgrupp i \(D_4\)?
% \end{problem}
% 
% \begin{problem}
%   Ge ett exempel på en icke-normal delgrupp till \(S_3\).
% \end{problem}
% 
% \begin{definition}
%   Låt \(H\) vara en normal delgrupp till gruppen \(G\). Mängden av alla sidoklasser \(M\) tillsammans med operatorn \((a  \star H)  \star (b \star H) = (a  \star b) H\) skapar \textit{kvotgruppen} till \(H\).
% \end{definition}
% 
% \begin{problem}
%   Vad är kvotgruppen av heltalen under addition med delgruppen av tal delbara med \(7\)?
% \end{problem}
% 
% \begin{problem}
%   Varför vill vi i definitionen av en kvotgrupp att \(H\) ska vara normal?
% \end{problem}
% 
% \begin{problem}
%   Vad skulle kvotgrupper kunna användas till?
% \end{problem}



\end{document}
