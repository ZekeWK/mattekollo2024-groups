\documentclass[11pt,fleqn]{book} % Default font size and left-justified equations

\input{structure.tex} % Insert the commands.tex file which contains the majority of the structure behind the template

\usepackage{etoolbox}
\makeatletter
\patchcmd{\chapter}{\if@openright\cleardoublepage\else\clearpage\fi}{}{}{}
\makeatother
\pagestyle{empty}
\setcounter{chapter}{3}

\begin{document}
  \renewcommand*\rmdefault{ppl}\normalfont\upshape

\chapter{Gruppteori 4. Lagrange's Teorem}
\kapiteldatum{24 juli}
\large
\thispagestyle{empty}

\begin{theoremeT} \textbf{(Lagrange's Teorem)} Storleken av en ändlig grupp delas av storleken av dess delgrupper.
\end{theoremeT} % Förtydliga

\begin{problem}
  Från förra lektionen vet vi att alla element i gruppen är i någon restklass till varje delgrupp, att två restklasser antingen är disjunkta eller samma samt att alla restklasser hara samma antal element. Bevisa Lagrange's Teorem.
\end{problem}

\begin{problem}
  Visa att om storleken av två delgrupper är relativt prima så är deras snitt \(\left\{e\right\}\).
\end{problem}

\begin{problem}
  Eulers sats inom talteori säger att för ett tal \(a\) och ett primtal \(p\) så är
  \(
    a^k \equiv 1 \mod p
  \)
  om \(k\) är antalet element mindre än \(a\) som är relativt primt med det. Bevisa Eulers sats!
  \\ \\
  Tips! Ett tal är relativt primt \(n\) om och endast om det är inverterbart modulo \(n\) . 
\end{problem}




\end{document}
