\documentclass[11pt,fleqn]{book} % Default font size and left-justified equations

\input{structure.tex} % Insert the commands.tex file which contains the majority of the structure behind the template

\usepackage{etoolbox}
\makeatletter
\patchcmd{\chapter}{\if@openright\cleardoublepage\else\clearpage\fi}{}{}{}
\makeatother
\pagestyle{empty}
\setcounter{chapter}{3}

\begin{document}
  \renewcommand*\rmdefault{ppl}\normalfont\upshape

\chapter{Gruppteori 4. Lagranges sats}
\kapiteldatum{24 juli}
\large
\thispagestyle{empty} 

\begin{definition} Låt \(H = \left\{h_1, h_2, \dots, h_n\right\}\) vara en delgrupp av \(G\). För varje element \(a\) i \(G\) definierar vi \textit{sidoklassen} \(a \star H = \left\{a  \star h_1, a \star h_2, \dots, a \star h_n\right\}\).
\end{definition}

\begin{problem}
  Låt \(M\) vara en delgrupp till \(G\) med \(k\) element. Hur många element har \(a  \star M\)?
\end{problem}

\begin{problem} 
	Finns det något element som inte är i någon sidoklass?
\end{problem}

\begin{problem}
	Visa att om två sidoklasser båda innehåller ett visst element så sammanfaller de.
\end{problem}

\begin{theoremeT} \textbf{(Lagranges sats)} Storleken av en ändlig grupp delas av storleken av dess delgrupper.
\end{theoremeT} % Förtydliga

\begin{problem}
  Bevisa Lagranges sats.
\end{problem}

\begin{problem}
  Hitta de två delgrupperna till \(\mathbb{Z} _{29}\).
\end{problem}

\begin{problem}
  Visa att alla grupper av primtals storlek är cykliska.
\end{problem}

\begin{problem}
  Visa att om storleken av två delgrupper är relativt prima så är deras snitt \(\left\{e\right\}\).
\end{problem}

\begin{problem}
 Visa att \(a ^{|G|} = e\) för elementet \(a\) i gruppen \(G\) med \(|G| \) stycken element.
\end{problem}

\begin{theoremeT} \textbf{(Eulers sats).} 
  För två positiva och relativt prima heltal \(a, n\) gäller att \(a ^{\varphi (n)} \equiv 1 \mod n\) där \(\phi (n)\) är antalet tal mindre än \(n\) som är relativt prima till \(n\).
\end{theoremeT}

\begin{problem}
  Bevisa Eulers sats.
\end{problem}

\subsection*{Extrauppgifter}
Jobba på uppgifterna från tidigare blad. Det finns även ett blad om gruppverkan och Burnsides lemma om man önskar.


\end{document}
