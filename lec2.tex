\documentclass[11pt,fleqn]{book} % Default font size and left-justified equations

\input{structure.tex} % Insert the commands.tex file which contains the majority of the structure behind the template

\usepackage{etoolbox}
\makeatletter
\patchcmd{\chapter}{\if@openright\cleardoublepage\else\clearpage\fi}{}{}{}
\makeatother
\pagestyle{empty}
\setcounter{chapter}{1}

\begin{document}
\renewcommand*\rmdefault{ppl}\normalfont\upshape

\chapter{Gruppteori 2. Isomorfier \& delgrupper}
\kapiteldatum{21 juli}
\large
\thispagestyle{empty}

\begin{definition} En \textit{isomorfi} \(\varphi \) mellan grupperna \(G_1\) och \(G_2\) är en kartläggning (funktion) så att:
  \begin{enumerate} % Omformulera!
    \item Varje element i \(G_2\) antas som värde av \(\varphi\) för exakt ett element i \(G_1\).
    \item För alla \(a, b\) i \(G_1\) gäller \(\varphi(a) \star \varphi (b) = \varphi(a  \star b)\).
    \item För identitetselementen \(e_1, e_2\) i \(G_1\) respektive \(G_2\) gäller att \(\varphi(e_1) = e_2\).
  \end{enumerate}
\end{definition}

\begin{problem}
  Gruppen \(G_1\) av ``klä om strumpan''-operationer (från förra lektionen), består av operationerna: ``gör inget'', ``byt fot'', ''vänd ut och in'', ``vänd ut och in och byt fot''. Gruppen \(G_2\) består istället av mängden \(\left\{e, a, b, c\right\}\) och operatorn \(\star\) som uppfyller multiplikationstabellen 
  { \center
  \begin{tabular}{c|cccc}
    \( \star \) & \(e\) & \(a\) & \(b\) & \(c\)  \\ \midrule
    \(e\) & \(e\) & \(a\) & \(b\) & \(c\)  \\
    \(a\) & \(a\) & \(e\) & \(c\) & \(b\) \\
    \(b\) & \(b\) & \(c\) & \(e\) & \(a\) \\
    \(c\) & \(c\) & \(b\) & \(a\) & \(e\) \\
  \end{tabular}\\}
  \vspace{2ex}
  \noindent Hitta en isomorfi mellan \(G_1\) och \(G_2\).
\end{problem}

\begin{problem}
  Visa att \(D_3\) är isomorf med permutationsgruppen \(S_3\), mängden av alla omordningar av 3 element.
\end{problem}

\begin{problem}
  Motivera att gruppen av rotationer av en kub är isomorf med permutationsgruppen \(S_4\). 
\end{problem}



% \begin{problem}
%   Låt \(G_1\) vara heltalen modulo 4 under addition och \(G_2\) vara rotationerna av en kvadrat som bevarar den.
%   \begin{enumerate}[label=\textbf{(\alph*)}]
%     \item Gör en additionstabell för \(G_1\). 
%     \item Gör en multiplikationstabell för \(G_2\). 
%     \item Hitta en isomorfi mellan grupperna.
%     \item Överensstämmmer de?
%   \end{enumerate} % TODO!
%   
% \end{problem}

% \begin{problem}
%   Låt \(\varphi (n) = 2n\). Är \(\varphi\) en isomorfi från heltalen under addition till de jämna heltalen under addition? Vilka andra sådana isomorfier finns det? 
% \end{problem}

% \begin{problem}
%   Är \(D_3\) isomorf med \(S_3\), mängden av alla permutationer (omordningar) av 3 element? Vad gäller för \(D_4\) och \(S_4\)?
% \end{problem}

\begin{definition}
  En grupp \(G\) kallas \textit{cyklisk} om det finns ett element \(a\) i \(G\) så att mängden till \(G\) är exakt potenserna till \(a\). 
\end{definition}

\begin{problem}\textbf{(Cykliska grupper).} Visa att heltalen modulo \(n\) under addition är cykliska.
\end{problem}

\begin{problem}
  Grupperna \(G_1, G_2\) kallas isomorfa om det finns en isomorfi mellan dem. Visa att om \(G_1\) är isomorf med \(G_2\) och \(G_2\) är isomorf med \(G_3\) så är \(G_1\) isomorf med \(G_3\).
\end{problem}

\begin{problem}\textbf{(Cykliska grupper).} Visa att alla cykliska grupper av storlek \(n\) är isomorfa med varandra och inga andra.
\end{problem}

% \begin{problem} \textbf{(Cykliska grupper).} Låt \(a\) vara ett element i \(G_1\) och \(\varphi \) vara en isomorfi från \(G_1\) till \(G_2\). Visa att \(a^n = e_1 \) är ekvivalent \(\varphi(a)^n = e_2\).
% \end{problem}

\begin{definition}
  Låt \(H = (M_H,  \star)\) och \(G = (M_G,  \star)\) vara grupper. \(H\) är en \textit{delgrupp} till \(G\) om \(M_H\) är en delmängd till \(M_G\).
\end{definition}

\begin{problem}
  Är mängden av alla \textbf{(a)} speglingar \textbf{(b)} rotationer en delgrupp av \(D_n\)?
\end{problem}

\begin{problem}
  Hitta 4 delgrupper till dragen på en Rubiks Kub som har olika storlek.
\end{problem}

\begin{problem}
  Låt \(\varphi \) vara en isomorfi från \(G_1\) till \(G_2\) och och \(H_1 = (\left\{a_1, \dots, a_n \right\},  \star )\) vara en delgrupp till \(G_1\). Visa att \(H_2 = ( \left\{\varphi (a_1), \dots, \varphi(a_n)\right\},  \star )\) är en delgrupp till \(G_2\).
\end{problem}

\begin{problem}
  Hitta en delgrupp av \(S_4\) som är isomorf med ``klä om strumpan''-gruppen.
\end{problem}


\subsection*{Extra uppgifter}


\begin{definition}
  För två grupper \(G_1, G_2\) definieras den \textit{direkta produkten} \(G_1 \times G_2\) som gruppen där:
  \begin{enumerate}
    \item Elementen är mängden av alla par \((a_1, a_2)\) där \(a_1, a_2\) är element i \(G_1\) respektive \(G_2\).
    \item Operatorn appliceras elementvist enligt \((a_1, a_2)  \star (b_1, b_2) = (a_1  \star b_1, a_2  \star b_2)\).
  \end{enumerate}
\end{definition}

\begin{problem} \textbf{(Direkta produkten).} 
  Låt \(G = \mathbb{Z} _4 \times \mathbb{Z} _5\) under addition. Vad är inversen till \((2, 0)\)? 
\end{problem}


\begin{problem} \textbf{(Direkta produkten).} 
  Låt \(\mathbb{Z}_n \) vara heltalen modulo \(n\). Är \textbf{(a)} \(\mathbb{Z} _2 \times \mathbb{Z} _2\) isomorf med \(\mathbb{Z} _4\) \textbf{(b)} \(\mathbb{Z} _2 \times \mathbb{Z} _3\) isomorf med \(\mathbb{Z} _6\)? % Förtydliga?
\end{problem}

\begin{problem} \textbf{(Direkta produkten).} 
  Visa att \(\mathbb{Z} _2 \times \mathbb{Z} _2\) under addition är isomorf med ''klä om strumpan``-gruppen.
\end{problem}



% \begin{problem} \textbf{(Direkta produkten).} 
%   Vilka krav ställs på \(n\) och \(m\) om \(\mathbb{Z} _{n  \cdot m} \sim \mathbb{Z} _n \times \mathbb{Z} _m\)?
% \end{problem}
% 
% \begin{problem} \textbf{(Direkta produkten).} 
%   Låt \(\varphi \) vara en isomorfi från \(\mathbb{Z} _ {2  \cdot 3  \cdot 5  \cdot 7}\) till \(\mathbb{Z} _2 \times \mathbb{Z} _3 \times \mathbb{Z} _5 \times \mathbb{Z} _7\). \textbf{(a)} Beräkna \(\varphi (11 ^{-1})\) där \(11 ^{-1}\) är den multiplikativa inversen till \(11\) i \(\mathbb{Z} _{2  \cdot 3  \cdot 5  \cdot 7}\). \textbf{(b)} Beskriv hur man skulle kunna beräkna \(11 ^{-1}\).
% \end{problem}





\begin{problem} % Change ordering perhaps?
  Är de rationella talen under addition isomorfa med de nollskilda rationella talen under multiplikation?
\end{problem}

\begin{problem}
  Är heltalen under addition isomorfa med de rationella talen under addition?
\end{problem}

\begin{problem}
  Är de rationella talen under addition isomorfa med de reella talen under addition?
\end{problem}

% \begin{problem} \textbf{(Direkta produkten).} 
%   Vilka krav ställs på \(n\) och \(m\) om \(\mathbb{Z} _{n  \cdot m} \sim \mathbb{Z} _n \times \mathbb{Z} _m\)?
% \end{problem}
% 
% \begin{problem} \textbf{(Direkta produkten).} 
%   Låt \(\varphi \) vara en isomorfi från \(\mathbb{Z} _ {2  \cdot 3  \cdot 5  \cdot 7}\) till \(\mathbb{Z} _2 \times \mathbb{Z} _3 \times \mathbb{Z} _5 \times \mathbb{Z} _7\). \textbf{(a)} Beräkna \(\varphi (11 ^{-1})\) där \(11 ^{-1}\) är den multiplikativa inversen till \(11\) i \(\mathbb{Z} _{2  \cdot 3  \cdot 5  \cdot 7}\). \textbf{(b)} Beskriv hur man skulle kunna beräkna \(11 ^{-1}\).
% \end{problem}

\end{document}
