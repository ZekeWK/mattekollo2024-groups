\documentclass[11pt,fleqn]{book} % Default font size and left-justified equations

\input{structure.tex} % Insert the commands.tex file which contains the majority of the structure behind the template

\usepackage{etoolbox}
\makeatletter
\patchcmd{\chapter}{\if@openright\cleardoublepage\else\clearpage\fi}{}{}{}
\makeatother
\pagestyle{empty}
\setcounter{chapter}{1}

\begin{document}
  \renewcommand*\rmdefault{ppl}\normalfont\upshape

\chapter{Gruppteori 2. Isomorfier}
\kapiteldatum{21 juli}
\large
\thispagestyle{empty}



\begin{definition} En \textit{isomorfi} \(\varphi \) mellan grupperna \(G_1\) och \(G_2\) är en kartläggning (funktion) så att:
  \begin{enumerate} % Omformulera!
    \item \textbf{Bijektiv}: Varje element i \(G_2\) antas som värde av \(\varphi\) för exakt ett element i \(G_1\).
    \item \textbf{Bevarar operatorn}: För alla \(a, b\) i \(G_1\) gäller \(\varphi(a) \star \varphi (b) = \varphi(a  \star b)\).
    \item \textbf{Bevarar identitet}: För identitetselementen \(e_1, e_2\) i \(G_1\) respektive \(G_2\) gäller att \(\varphi(e_1) = e_2\).
  \end{enumerate}
\end{definition}

\begin{problem}
  Låt \(\varphi (n) = 2n\).
  \begin{enumerate}[label=\textbf{\alph*)} ]
    \item Är \(\varphi\) en isomorfi från heltalen under addition till de jämna heltalen under addition?
    \item Finns det andra isomorfier som uppfyller (a)? 
    \item Är \(\varphi\) en isomorfi från heltalen under multiplikation till de jämna heltalen under multiplikation?
  \end{enumerate}
\end{problem}

\begin{problem}
  Finns det en isomorfi mellan \(D_n\) och heltalen modulo \(2n\) under addition om \textbf{(a) \(n = 3\) } \textbf{(b)} \(n > 3\)?
\end{problem}

\begin{problem}
  Två grupper \(G_1, G_2\) kallas isomorfa, \(G_1 \sim G_2\),  om det finns en isomorfi mellan dem. Visa att om \(G_1\) är isomorf med \(G_2\) och \(G_2\) är isomorf med \(G_3\) så är \(G_1\) isomorf med \(G_3\).
\end{problem}

\begin{problem}
  Är \(D_3\) isomorf med \(S_3\), mängden av alla permutationer (omordningar) av 3 element? Vad gäller för \(D_4\) och \(S_4\)?
\end{problem}

\begin{problem} En grupp \(G\) kallas cyklisk om det finns ett element \(a\) så att \(G\) är exakt potenserna till \(a\). 
  \begin{enumerate}[label=\textbf{\alph*)} ]
    \item Visa att heltalen under addition är cykliska.
    \item Visa att alla cykliska grupper av storlek \(n\) är isomorfa.
  \end{enumerate}
  
\end{problem}

\begin{problem}
  För två grupper \(G_1, G_2\) definieras den direkta produkten \(G_1 \times G_2\) som gruppen där:
  \begin{enumerate}
    \item Elementen är mängden av alla par \((a_1, a_2)\) där \(a_1, a_2\) är element i \(G_1\) respektive \(G_2\).
    \item Operatorn appliceras elementvist enligt \((a_1, a_2)  \star (b_1, b_2) = (a_1  \star b_1, a_2  \star b_2)\).
  \end{enumerate}
  Är \textbf{(a)} \(\mathbb{Z} _2 \times \mathbb{Z} _2 \sim \mathbb{Z} _4\) \textbf{(b)} \(\mathbb{Z} _2 \times \mathbb{Z} _3 \sim \mathbb{Z} _6\)? % Förtydliga?
\end{problem}

\begin{problem} % Change ordering perhaps?
  Är de rationella talen under addition ismorfa med de nollskilda rationella talen under multiplikation?
\end{problem}

\begin{problem}
  Är heltalen under addition isomorfa med de rationella talen under addition?
\end{problem}

\begin{problem}
  Är de rationella talen under addition isomorfa med de reella talen under addition?
\end{problem}

\begin{problem}
  Är de komplexa talen under addition isomorfa med \(\mathbb{R} ^2\) (punkter i reella talplanet) under addition?
\end{problem}

\begin{problem}
  Hur kan man definiera produkten på \(\mathbb{R} ^2\) för att de ska vara ismorfa med de komplexa talen under multiplikation?
\end{problem}


% TODO! More difficult problems

\end{document}
